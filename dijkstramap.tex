\documentclass[12pt,a4paper]{article}
\setlength{\topmargin}{-5mm}
\setlength{\textheight}{244mm}
\setlength{\oddsidemargin}{0mm}
\setlength{\textwidth}{165mm}

\usepackage{fancyvrb}
\DefineVerbatimEnvironment{verbatim}{Verbatim}{xleftmargin=4mm}

\usepackage{lmodern}
\usepackage[T1]{fontenc}
\usepackage{pxfonts}
\usepackage{textcomp}
\usepackage{tabu}

\title{Dijkstra Maps in Roguelikes}
\author{Jeremy Mates}
\date{September 23, 2018}

\usepackage{hyperref}
\hypersetup{pdfauthor={Jeremy Mates},pdftitle={Dijkstra Maps in Roguelikes}}

\begin{document}
\bibliographystyle{plainnat}
\maketitle

\setlength{\parindent}{0pt}

\section*{Simple Pathfinding}

Consider the following map, wherein there are three velociraptors
\texttt{R} surrounding the player \texttt{@} on an open
field\cite{xkcd135}. How can the raptors reach the player?

\begin{verbatim}
......R
.R.....
.......
...@...
.......
.......
..R....
\end{verbatim}

One solution to this problem is the so-called Dijkstra Map\cite{tipodm}.
A map this small can easily be worked out on graph paper; set the
location of the player to the value \texttt{0}. Then in the squares
adjacent to that point write the next highest number, \texttt{1}.

\begin{verbatim}
.......
.......
...1...
..101..
...1...
.......
.......
\end{verbatim}

And so forth until the map is filled. This is a Dijkstra Map.

\begin{verbatim}
6543456
5432345
4321234
3210123
4321234
5432345
6543456
\end{verbatim}

Note that diagonal moves were not considered\textendash only those
points North South East or West of the given square. This has
ramifications on more complicated maps that contain obstacles.
Meanwhile, each velociraptor can use this map to find the player by
walking ``downhill'' to \texttt{0}, or can flee by picking a cell with a
higher value.

\section*{Obstacles}

Many maps will be more complicated than the above and will contain
obstacles, typically walls, or lava, or there may be monsters that can
pass through walls but not across holy water\textendash spectres
perhaps\textendash or anything else you can imagine. Let us for now
only consider physical walls, represented using \texttt{\symbol{35}} as
is typical for roguelikes.

\begin{verbatim}
R....
.###.
.#@..
.#.##
.#.#.
...#.
\end{verbatim}

A Dijkstra Map for this might look like

\vskip 1em%
\begin{tabu} spread 0pt{|X|X|X|X|X|} \hline
8 & 7 & 6 & 5 & 4\\\hline
9 & -1 & -1 & -1 & 3\\\hline
8 & -1 & 0 & 1 & 2\\\hline
7 & -1 & 1 & -1 & -1\\\hline
6 & -1 & 2 & -1 & ?\\\hline
5 & 4 & 3 & -1 & ?\\\hline
\end{tabu}
\vskip 1em%

where the impassable walls are represented by \texttt{-1}. One may use
\texttt{Inf} or an object for such cells, but I assume here an integer
or fixnum array. Another option would be to leave walls at the maximum
integer or \texttt{MOST-POSITIVE-FIXNUM} value and to ignore them while
calculating the costs. Note the two squares lower right that have no
path to the player; these after calculation will typically remain at the
maximum integer value the field was filled with before the pathfinding
pass. This map by the way was made with my \texttt{Game::DijkstraMap}
module\cite{gdm}.

\begin{verbatim}
#!/usr/bin/env perl
use strict;
use warnings;
use Game::DijkstraMap;
my $dm = Game::DijkstraMap->new( str2map => <<'EOM' );
.....
.###.
.#x..
.#.##
.#.#.
...#.
EOM
print $dm->to_tsv;
\end{verbatim}

Pathfinding on this map must ignore \texttt{-1} and look for values
equal to or greater than \texttt{0} when routing to a goal. Here, the
raptor should move horizontally to reach the player in as few moves
as possible.

\section*{Multiple Goals}

A Dijkstra Map may contain multiple goals; pathfinding will find the (or
a) nearest goal. Crabs \texttt{c} for example may wish to retreat to
water \texttt{\symbol{126}} when threatened by the player.

\begin{verbatim}
..~~......~.....
.~~~~......~~...
...~~..c..~~....
..........c.....
...~~...........
..~~...c........
...~~...........
.....~.........@
\end{verbatim}

\vskip 1em%
\begin{tabu} spread 0pt{|X|X|X|X|X|X|X|X|X|X|X|X|X|X|X|X|} \hline
2 & 1 & 0 & 0 & 1 & 2 & 3 & 4 & 2 & 1 & 0 & 1 & 1 & 2 & 3 & 4\\\hline
1 & 0 & 0 & 0 & 0 & 1 & 2 & 3 & 4 & 2 & 1 & 0 & 0 & 1 & 2 & 3\\\hline
2 & 1 & 1 & 0 & 0 & 1 & 2 & 3 & 2 & 1 & 0 & 0 & 1 & 2 & 3 & 4\\\hline
3 & 2 & 2 & 1 & 1 & 2 & 3 & 4 & 3 & 2 & 1 & 1 & 2 & 3 & 4 & 5\\\hline
4 & 3 & 1 & 0 & 0 & 1 & 2 & 3 & 4 & 3 & 2 & 2 & 3 & 4 & 5 & 6\\\hline
2 & 1 & 0 & 0 & 1 & 2 & 3 & 4 & 5 & 4 & 3 & 3 & 4 & 5 & 6 & 7\\\hline
3 & 2 & 1 & 0 & 0 & 1 & 2 & 3 & 4 & 5 & 4 & 4 & 5 & 6 & 7 & 8\\\hline
4 & 3 & 2 & 1 & 1 & 0 & 1 & 2 & 3 & 4 & 5 & 5 & 6 & 7 & 8 & 9\\\hline
\end{tabu}
\vskip 1em%

This map however does not consider other crabs\textendash can two occupy
the same square, or will one need to find a longer route to a free water
cell?\textendash nor the influence of the player; a monster may
realistically have multiple desires: get to water while also avoiding
the player. This requires a combination of Dijkstra Maps, the inverse
distance to the player combined with the above water map. (This is an
area I and my code need to level up a bit more in, so I will say no more
about it.)

\section*{The Diagonal}

Costs in the above maps have been done without consideration for
diagonal moves. Consider the following map, where \texttt{x} is the
goal.

\begin{verbatim}
@#..
#...
...x
\end{verbatim}

\vskip 1em%
\begin{tabu} spread 0pt{|X|X|X|X|} \hline
? & -1 & 3 & 2\\\hline
-1 & 3 & 2 & 1\\\hline
3 & 2 & 1 & 0\\\hline
\end{tabu}
\vskip 1em%

The player here cannot pathfind to the goal as the diagonal move was not
considered by the 4-way algorithm that only consults cells North South
East and West. Various roguelikes (Angband, Dungeon Crawl Stone Soup)
permit such diagonal moves, so will need to use an 8-way algorithm when
constructing a Dijkstra Map. Other roguelikes (Brogue, POWDER) may deny
such diagonal moves so can use the 4-way algorithm. This choice also
influences level design. 4-way and 8-way roguelikes require rather
different diagonal corridors:

\begin{verbatim}
8-way    @####   4-way    @.###
corridor #.###   corridor #..##
         ##.##            ##.##
         ###x#            ##.x#
         #####            #####
\end{verbatim}

Brogue is complicated in that it allows some diagonal moves. The choices
for a two dimensional roguelike are pure 8-way, pure 4-way, or hybrid
models somewhere between those two extremes. A 4-way Dijkstra Map
algorithm can be used with 8-way motion provided 4-way moves are
possible to everywhere that must be reached. Diagonal moves in such a
case exist as shortcuts\textendash moving diagonally along the above
4-way corridor (which Brogue would not permit, nor POWDER unless one is
ploymorphed into a grid bug).

\section*{Diagonal Maps}

8-way maps typically assign equal costs to all adjacent squares. The
original raptor map thus looks like:

\begin{verbatim}
......R  3333333
.R.....  3222223
.......  3211123
...@...  3210123
.......  3211123
.......  3222223
..R....  3333333
\end{verbatim}

This while traditional for roguelikes is not actually correct; diagonals
under euclidean geometry should instead use $\sqrt{x^2 + y^2}$ or
$\sqrt{2}$ instead of \texttt{1} for the closest diagonal. Various
roguelikes are actually non-euclidean: Brogue and Dungeon Crawl Stone
Soup apply the same cost to a move in any direction, diagonal or
otherwise. Anyways! Our original diagonal map that stumped the 4-way map
under (non-euclidean) 8-way is:

\begin{verbatim}
@#..
#...
...x
\end{verbatim}

\vskip 1em%
\begin{tabu} spread 0pt{|X|X|X|X|} \hline
3 & -1 & 2 & 2\\\hline
-1 & 2 & 1 & 1\\\hline
3 & 2 & 1 & 0\\\hline
\end{tabu}
\vskip 1em%

And the player can path to the goal.

\section*{Not Just for Animates}

Dijkstra Map can be used for other purposes. Given a map with two rooms
in it, how to connect them?

\begin{verbatim}
####################
###########XXXXXX###
###########X....X###
###########X....X###
#XXXXXX####XXXXXX###
#X....X#############
#X....X#############
#X....X#############
#XXXXXX#############
####################
\end{verbatim}

Place a goal anywhere in the wall space, and then considering only walls
\texttt{\symbol{35}} as passable construct a Dijkstra Map with the goal
as the destination, here shown partially complete.

\begin{verbatim}
           XXXXXX   
           X....X8  
     DCBA98X....X7  
 XXXXXXA987XXXXXX6  
 X....X98765432345  
 X....X87654321234  
 X....X76543210123  
 XXXXXX 765432123   
\end{verbatim}

Then pick a random door location in each room, and pathfind from each
door to the goal, drawing a corridor along the way.

\begin{verbatim}
####################
###########XXXXXX###
###########X....+.##
#####...###X....X.##
#XXXX+X.###XXXXXX.##
#X....X...#######.##
#X....X##.........##
#X....X#######,.####
#XXXXXX#############
####################
\end{verbatim}

This will run into complications if rooms are adjacent or worse two
rooms block a third from reaching a goal (there should likely be one
goal per unconnected wall space in a map) though there are various
solutions to these challenges, such as pathfinding through rooms or
using additional code to find and link up adjacent rooms.

\section*{Super Dimensional Dijkstra Maps}

We need not confine ourselves to two, or even three dimensions;
Dijkstra Maps can be built in arbitrary numbers of dimensions
(memory requirements, implementation demands, and sanity
permitting). The following is a four-dimensional map with an
implementation\cite{crazylisp} that does not consider diagonal
moves legal.

\begin{verbatim}
% sbcl --noinform --load dijkstramap.lisp
* (setf *dimap-cost-max* 99)

99
* (defparameter level
    (make-array '(3 3 3 3)
                 :initial-contents
                 '((((99 -1 99) (-1 99 -1) (99 -1 99))
                    ((-1 99 99) (99 99 99) (99 99 -1))
                    ((99 99 99) (99 99 99) (99 99 99)))
                   (((-1 99 99) (99 99 99) (99 99 -1))
                    ((99 99 99) (99  0 99) (99 99 99))
                    ((99 99 99) (99 99 99) (99 99 99)))
                   (((99 99 99) (99 99 99) (99 99 99))
                    ((99 99 99) (99 99 99) (99 99 99))
                    ((99 99 99) (99 99 99) (99 99 99))))))

LEVEL
* (dimap-calc level)

4
* level

#4A((((99 -1 4) (-1 2 -1) (4 -1 99))
     ((-1 2 3) (2 1 2) (3 2 -1))
     ((4 3 4) (3 2 3) (4 3 4)))
    (((-1 2 3) (2 1 2) (3 2 -1))
     ((2 1 2) (1 0 1) (2 1 2))
     ((3 2 3) (2 1 2) (3 2 3)))
    (((4 3 4) (3 2 3) (4 3 4))
     ((3 2 3) (2 1 2) (3 2 3))
     ((4 3 4) (3 2 3) (4 3 4))))
*
\end{verbatim}

Productive uses for such maps are left as an exercise to the reader.

\clearpage
\bibliography{references}
\end{document}
