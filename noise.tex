\documentclass[12pt,a4paper]{article}
\setlength{\topmargin}{-5mm}
\setlength{\textheight}{244mm}
\setlength{\oddsidemargin}{0mm}
\setlength{\textwidth}{165mm}

\usepackage{fancyvrb}
\DefineVerbatimEnvironment{verbatim}{Verbatim}{xleftmargin=4mm}

\usepackage{lmodern}
\usepackage[T1]{fontenc}
\usepackage{pxfonts}
\usepackage{textcomp}

\title{``white'' versus ``brown'' noise in roguelike maps}
\author{Jeremy Mates}
\date{May 13, 2018}

\usepackage{hyperref}
\hypersetup{pdfauthor={Jeremy Mates},pdftitle={``white'' versus ``brown'' noise in roguelike maps}}

\begin{document}
\bibliographystyle{plainnat}
\maketitle

\setlength{\parindent}{0pt}

\section*{White noise}

is simply random\cite{bulmer2000music}. In the context of a roguelike
given a grid of arbitrary size white noise may be achieved by placing
objects randomly on the grid. The file \texttt{noise.lisp} details the
complete code; only snippets will be shown here.

\begin{verbatim}
(defparameter *rows*  20)
(defparameter *cols*  72)

(defparameter *board* (make-array (list *rows* *cols*) ...)

(dotimes (n 100))
  (setf (aref *board* (random *rows*) (random *cols*)) #\x)))
\end{verbatim}

This produces no apparent pattern (and a human may dismiss it as ``oh,
random'').

\begin{verbatim}
...xxx...............................x..x......x..............x.........
.............x....x...............x................................x....
.........................xx...............x..........x.........x........
....x................x.....x.x............x.........x.....x..........x..
............xx...............x..........................................
..x...............................................................x..x..
....x.................................................................x.
............................................x..............xx...........
...............x............x....................................x......
......x...........x...x.......x............x.................x..........
...............x.............................x.....................x....
.x....x.................................xx..x..............x.....x......
......................x....x.........x..........x.x.................x...
.x.x................................xx...xx.x.........x...x.............
.........................x..............................................
x....x.....................................x.....x.............x........
......x......................x..x..........................x............
...............................................x........................
..............xxx......x..............x......x...........x............x.
............x......x.........x.......x.....x...................x......x.
\end{verbatim}

\section*{Brown noise}

by contrast will locate each new object relative to the previous object.
Care must be taken that the objects do not walk off of the grid (unless
they wrap around) and that there is no bias (unless such is desired)
that makes the walk favor a particular direction.

\begin{verbatim}
(defun if-legal (new max)
  (and (>= new 0) (< new max) new))

(defun nearby (x max range)
  (do ((new nil))
    ((numberp new) new)
    (setf new (if-legal (+ x (- (random range)
                                (truncate (/ range 2)))) max))))

(let ((r (random *rows*)) (c (random *cols*)))
  (dotimes (n 100)
    (setf (aref *board* r c) #\x)
    (setf r (nearby r *rows* 5))
    (setf c (nearby c *cols* 9)))))
\end{verbatim}

A notable feature here is that the randomization is different for the
rows and columns; rows move $\pm 2$ and columns $\pm 4$. Neither this
nor the previous implementation care if an object overwrites another.

\begin{verbatim}
......................................xx.x....x.......x.................
..................................x..xx..x...xx.x.x.....x...............
.......................................xxx.xx..........x................
.....................................xx.xx.................xx...........
......................................x..x..x..........x..x.............
...........................................x............................
...............................................x......xxx...............
.............................................xxxx.......................
............................................x..x.x.x......x.............
...........................................x.x.xx...x...................
............................................x.......x.........x.........
...........................................x.xx.......x........x........
....................................................xx..................
....................................................x.x............x....
.................................................x................xx....
...................................................x...x..........x...x.
........................................................x.x...........x.
.....................................................x........x..x.x....
........................................................x..xx...........
.........................................................x...xxxx.x.....
\end{verbatim}

This is obviously quite different from the white noise; moreover,
repeated runs of the white noise version will look more or less the same
while the brown noise will vary greatly depending on the starting point
and exact sequence of random values.

With multiple object types the differences between white and brown noise
can be made most notable; instead of \texttt{x} the following place
plants, fungus, and rocks.

\begin{verbatim}
............#.......P.....................f..f......#P..f........P..#..P
.......#...P...........#.........P....P..PP........f......P..#...P..#..f
.....P..P.P..........#f.f.f...f............#....#P...#..f..#............
f.P.P..#..f#...#.....P........#....#....f.P.fP...P....#f......f.f#.f....
...#.P.f...#.P.....P............f......P...............f.f...f..........
#...................Pf.....................#.....#...f.f.P.........f...P
.....#................P.f#...#..f.............#...#.f.#f...f......f....P
..P....Pf........ff.................f#.........ff..##.f........#........
.P.....#..#...P..........f.....f.......f.P...#ff.#..#.#.............f...
..f.P...#f..........fP.....f............#......fP....#..................
...f......P....f.....f.P...#....P...........P.P..f..P#.......P..........
....#.f#........#......f..f...fP......f.....P..#.#........#..##......#.P
#ff......#......#................f.................#....................
.....P........@.......#.P......P....P..f......#...........#.............
.#f#..#..........f....P..f....P...P.....P.P.P...P..P....#.......f.....#.
.........P...f......P....##....P#......#....P..#.#......fP..........#...
#f....f.....#.P.........f#...#..f#P...#........#............PP....f....#
.P...................f.#......f.....P..P.f...f.#....###....f........f..f
.PP.#PP.#..............##..#P.#...................................fP..#.
.#...#...................#..P....f...........f.......#.P...P....P.f.P..f
\end{verbatim}

and a brown noise version

\begin{verbatim}
......f......ff...f.ff........#...#####.................................
......f.....ff.......##............##.#.#.#.............................
...f...f..f.....f....#..#..#...######.#..##.............................
..........f.......#....#...#..##....###.#...............................
f...f....ff.......#......#...###.##.##.##...............................
......f...........#...#..#.#....#.#.#.................f.................
.f..................P...#..#.##..#.#####..#.............................
..................P....P.......###....##...#.........f..................
.P.P.f..P.....P.......P..P..##..#.#.....#................f..............
.P..P.....P..............#..#.##.......#...............f.f.f............
....P..fP..........P.P.PP.P###.............#............f..f............
..P...P.PP.fP....PP..P........#.............................f...........
....PP.f..f...@...f..Pf.....PP..............ff.f..............f.........
...PP..PP...f.....P..P.Pf.................f.f.f..........f..f..f........
P.P...P........f...f.f...f..............f........fff.f..................
.P.PPPP.........fP...f...f..f....ff.....f.f.f.f..f.fff..................
..PPP...P....................f......f.....f.f.f.ff.f..f.................
PPPPPP...P......P..PP......f..........f.f.f.......f..f..................
PPPPPPP.PP..............................f.f...f..ff...f.................
P.P.PPP.....P..............................fff..........................
\end{verbatim}

Other complications are possible under brown noise. The
randomization rules could vary depending on the object, or could
vary over time\textendash close randomization punctuated by the
occasional long leap would be an obvious thing to try (this may
approach ``pink noise'').

\section*{Evolved landscapes}

may in part replicate the real world; a first pass could use white
noise to distribute a random spread of colonizing species, and
subsequent passes use brown noise to place secondary growth. One could
also include rules for shade, nutrients, etc. though a simple mix of
white and brown noise will likely produce acceptable results. The
following used 3\% plants (white noise), 10\% fungus (brown noise), and
20\% rocks (brown noise).

\begin{verbatim}
..P...........................P.....P..f.Pfff.ff..f#f#f..#.fff.ff...ff.P
.............................................ff.ffff#fffff..f#f..##..#fP
................P...........P......P....ff......f.f.f#..ff.f..#..#.#...#
..............P.............................fP.ffff.#f#fffff...#.#####..
..........................P........................f#..##f#.#f....###..#
.........................P..P.................f..f#..#.#f#f#P.#.####.#..
.P.................P.................................ff...###f##..#.##..
.........................PP..........................#.#.####.###.......
...................................................P#..##.####.##.###.#.
..............@....................P..........#........##.#####.##......
..P.....P...........................P......##.......#.#.###.....###..#..
..........P........................P...P.#......#######.#...#.#.#..#...#
..............P..........P...................##.#.########.##.###.#.#.#.
...........................P..........#...#.....###.#.##.#..####.#......
\end{verbatim}

Cleanup passes may be required to sculpt the random results; look at
cellular automata algorithms for examples. Pathfinding may also need to
be done if an impassible map is not desired: hunting trails could be
carved out to join areas together. The real world does have various
amounts of impassable terrain; having some inaccessible areas may add
realism or interest to a game. (Note that while it may warm your black
heart to have \textit{The +17 Grand Scythe of Über Slaying} generate in
an inaccessible area players tend to be critically lacking in humor over
such quirks of the RNG\ldots)

\hskip 1.5em%
An improvement might be to generate random numbers of fungus and rocks
from random starting points, and using different rules for how much the
fungus and rocks can move. Code that does this can be found in
\texttt{noise.lisp}.

\begin{verbatim}
........#.......##...fPf..........f....P.........f.f.####.f.fPf.f...f...
.......#.#.#P..####..f#ff...f.f...................##..##..f...f......fff
.P...#.####......#.###.#fff.P.....PP.........fPff..##.......ff..f..f..ff
......###...........###.#.f.f.f.........ff.f#.....f#.ff...fff..ffff....f
......#...........P..###...................##.f.......f.ff.f......f.f...
................f...f##ff##.f........f...ff..#f......f.ff....f.f.f.f....
...#..........@...ff###.###..f.....f...f..f...f...f...f.ff.Pff..f...#..#
....####..........####..####f..fffff..f.ff.fff...f.........f...f..f.####
..###.###.......P....#...#f#f.f.......f...f..........#..f.ffff....f..###
..##.####.##.............#.##f..f......f.ff.........#.#..f.ff.fff..f####
..##..#.########....ff.##f###f.f.f.....f....f...P###..##...ff.f..ff#f#f.
...#.....#####.#.f..f#####.##f.ff..fff.f.#f.....###...##.ff.fffffff####.
.P..#.#fff.##ff......#######...f..f.f...P#.#...........###.f.ff..f#f###f
....##.###ff...........f.####..fff....f..####.P......ff###ff.f.f..f#f.#.
\end{verbatim}

\section*{Vaults}

enclose a particular amount of space with a particular structure; the
structures or rooms may be pre-designed or could be generated by some
algorithm. Either the vault or the noise could be drawn first,
depending on whether a definite vault structure or a ruin now overrun
by jungle is necessary. Some vaults may designate what points are
available to be filled in; these would be used to mask where the noise
is allowed to appear.

\begin{verbatim}
##.####    white ##.####    brown ##.####
##...##    noise ##.f.##    noise ##.f.##
#.....#     fill #...f.#     fill #.....#
#......          #....f.          #.f..f.
#.....#          #.....#          #.....#
##...##          ##f..##          ##.f.##
#######          #######          #######
\end{verbatim}

White noise will benefit smaller vaults where there is not enough space
available for a pattern to emerge. The above both used a 20\% fill (some
fungus were drawn under the walls) of which the white noise is much more
efficient to generate. If a human cannot tell the difference, use the
less expensive function.

\section*{Other improvements}

would be to where necessary maintain a count of objects placed; this
count could be used to ensure that the field has a suitable texture:
loop placing different types of objects, check the object count, repeat
if the output is still too sparse. This will be necessary if routines
with a high degree of variance are used (a 10\% chance of ending itself
\texttt{decay} function usually exits well before 20 iterations, until
it does not\ldots) or if the object placing functions only place a
fraction of the total required.

\hskip 1.5em%
The code used here does not check if a square is already occupied by
something; these checks will be necessary if there are structural
elements that must not change, or skipped if ornamental lichen are being
placed on an otherwise empty field.

\hskip 1.5em%
``Pink noise'' was alluded to above; this might be an interesting
line of study. For more information, see the

\bibliography{references}

\end{document}
